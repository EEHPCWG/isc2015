\documentclass{llncs}
\usepackage[none]{hyphenat}
\usepackage{graphicx}
\usepackage{url}
\sloppy

\title{Supercomputing Centers and Electricity Service Providers: A Geographically Distributed Perspective on Demand Management in Europe and the United States}
\author{Tapasya Patki\inst{1} \and Natalie Bates\inst{2} \and Girish Ghatikar\inst{3} \and Anders Clausen\inst{4} \and Sonja Klingert\inst{5} \and Ghaleb Abdulla\inst{1} \and Mehdi Sheikhalishahi\inst{6}}
\institute{Lawrence Livermore National Laboratory \and Energy Efficient High Performance Computing Working Group \and GreenLots \and University of Southern Denmark \and The University of Mannheim \and Create-Net \\ Contact Email: patki1@llnl.gov}
%\author{Natalie Bates and Tapasya Patki \\ Demand Response Team, Energy-Efficient HPC Working Group}
%\date{\today}
%
%\setlength{\topmargin}{-15mm}
%\setlength{\textwidth}{6.5in}
%\setlength{\oddsidemargin}{0mm}
%\setlength{\textheight}{8.5in}
%\setlength{\footskip}{1.5in}

\begin{document}
%\fontsize{10}{15}
%\selectfont
\maketitle

\abstract{
Supercomputing Centers (SCs) have high and variable power demands, which increase the challenges of the Electricity Service Providers (ESPs) with regards to efficient electricity distribution and reliable grid operation. High penetration of renewable energy generation further exacerbates this problem. In order to develop a symbiotic relationship between the SCs and their ESPs and to support effective power management at all levels, it is critical to understand and analyze how the existing relationships were formed and how these are expected to evolve. \\
In this paper, we first present results from a detailed, quantitative survey-based analysis and compare the perspectives of the European grid and SCs to the ones of the United States (US). We then show that contrary to the expectation, SCs in the US are more open toward cooperating and developing demand-management strategies with their ESPs. In order to validate this result and to enable a thorough comparative study, we also conduct a qualitative analysis by interviewing three large-scale, geographically-distributed sites: Oak Ridge National Laboratory (ORNL), Lawrence Livermore National Laboratory (LLNL), and the Leibniz Supercomputing Center (LRZ). We conclude that perspectives on demand management are dependent on the electricity market and pricing in the geographical region and on the degree of control that a particular SC has in terms of power-purchase negotiation.}

\newpage

\section{Introduction}

Current Supercomputing Centers (SCs) for High-Performance Computing (HPC) with peta-scale capabilities have high power demands, with peak requirements of over 30 MW and fluctuations of a few megawatts. This trend is expected to continue in the future as we push the limits of supercomputing further. As a result, Electricity Service Providers (ESPs) for such SCs need to support efficient electricity generation, transmission and distribution along with reliable grid operation. ESPs today already face reliability concerns for accommodating megawatt-level fluctuations from SCs and often require HPC client sites to forecast their electricity use. The acceptance and proliferation of renewable sources of energy further adds to the variability at the electricity generation end, making grid reliability even more challenging. A tighter integration and open communication between ESPs and their client SCs is thus critical as we proceed toward the next generation of supercomputing. \\

At present, most ESP-SC relationships are linear and unidirectional. Power is typically generated, distributed and delivered to customer sites without direct or active involvement, and most electricity pricing contracts are negotiated without any communication requirements. Going forward, however, it is expected that a multi-directional relationship will evolve between the ESPs and SCs.  Communication and control will flow from end-customers to one or more of the electricity generation and distribution entities, and contract terms will enforce stringent usage requirements. The cloud and data center providers, such as Google, have already started to anticipate this multi-directional relationship and are taking advantage of this changing landscape.  For example, Google's response suggests vertical integration, especially with Google's Energy Subsidiary which gives Google the right to sell energy within the United States. (need citation). Another example is the SmartGrid initiative \cite{SmartGrid} by the U.S. Department of Energy, which is making electricity delivery faster and more efficient by involving customers, adjusting to dynamic demands, and by providing automated solutions and quick responses to remote locations. Techniques involved in establishing such multi-directional relationships are referred to as \emph{demand management} (DM) techniques. The Energy-Efficient High-Performance Computing Working Group (EE HPC WG) seeks to analyze the impact of similar DM techniques for SCs with HPC workloads and their ESPs. \\

In our previous work, we focused on understanding how ESPs and SCs can work together to improve energy efficiency by surveying large-scale SCs in the United States \cite{BatesESP}. We developed a questionnaire and surveyed 11 sites. We noted that none of the SCs are working directly with their ESPs for demand management. Our main conclusion from this work was that SCs in the United States were interested in a tighter integration with their ESPs, but a business case for the same had not been demonstrated. In this work, we expand our analysis to include European SCs. We accomplished this by extending the aforementioned questionnaire and quantitatively surveying the European SCs. Nine out of the sixteen SCs that we contacted responded to the questionnaire. All except one of these sites were in Top 50 supercomputers in the world \cite{Top500}. \\

The main motivation for our geographical study lies in the way electricity is priced. In Europe, electricity is more expensive and is subject to more variability because of the use of renewable sources of energy. Additionally, the SCs in both geographical regions have different maximum power demands. For example, in the United States, four of the SCs we surveyed had HPC workloads of 10 MW or more. The remaining SCs  in the United States as well as all the SCs in Europe had workloads of 5 MW of less. \\

Overall, we expected that the European SCs will be more tightly integrated with their ESPs because of the higher prices and more extensive use of renewables in Europe.  Contrary to our expectations, however, we found that the United States shows more interest in responding to requests from their ESPs than Europe. The four SCs that needed 10MW or more had active communication channels with their ESPs about responding to grid requests.  None of the SCs in Europe had similar relationship with their ESPs. In this work, we present these results and analyze the differences in across the two geographies that may have led to this result. We first present results from our quantitative survey from 9 SC sites, and then conduct a detailed qualitative analysis for three major SCs: Oak Ridge National Laboratory (ORNL), Lawrence Livermore National Laboratory (LLNL), and Leibniz Supercomputing Center (LRZ). The key goal of the qualitative analysis is to delve deeper into the electricity pricing structures as well as the available incentives for a tighter integration, and to understand what motivates the existing relationship between SCs and their ESPs. \\

Section \ref{strategies} presents an overview of our demand management techniques and motivates the need for an open multi-directional relationship between SCs and their ESPs. Section \ref{res} discusses the quantitative results from the questionnaire. Section \ref{intws} presents details of our site-specific interviews and a qualitative analysis of the demand management options available to these sites. In Section \ref{comm}, we discuss our key findings and highlight some of the comments we received from our respondents. Section \ref{relwork} presents related work, adn Section \ref{summary} concludes this article.

\section{Strategies, Programs and Methods}
\label{spm}
In this subsection, we present the results of communication between SCs and their ESPs and the feasibility of a tighter integration. We define \emph{strategies} as power management techniques used by SCs to manage power. Strategies may or may not improve energy efficiency. For example, \emph{Load Migration} is a strategy that SCs may use in response to an ESP's request, and while it helps manage power effectively, it does not impact the energy efficiency of the site. On the other hand, fine-grained power management techniques, such as using node-level power capping, or better job scheduling algorithms are likely to improve energy efficiency but may not be as useful in response to an ESP request. Almost all sites employ some power management strategies, especially the ones involving lighting, temperature, cooling, fine-grain power management and job scheduling. There is moderate interest in grid integration strategies in the United States, and low interest in the same in Europe. From the point of view of SCs, strategies such as cutting jobs or load migration have little interest. 

\emph{Programs} are incentives offered by ESPs to their customers and to SCs in order to motivate them to help balance the electrical grid. Common examples include peak shedding, peak shifting and dynamic pricing. From our questionnaire, we concluded that neither European nor the United States sites are engaged with peak shedding, peak shifting or dynamic pricing programs at present. More sites in the United States have communicated with their ESPs regarding these programs. While both European and United States SCs are interested in dynamic pricing, there is mixed interest in peak shedding and peak shifting. The European sites are more interested in peak shedding than peak shifting, but the United States sites are more interested in peak shifting. 

We also asked our European respondents to indicate what might motivate them to communicate with their ESPs. The results are shown in Table \ref{fig:table2}. As can be noted from this figure, the main motivators are the financial incentives and the desire to be ``good citizens.''

\begin{figure}
\begin{center}
\includegraphics[scale=0.5]{figs/Table2.jpg}
\caption{Motivation}
\label{fig:table2}
\end{center}
\end{figure}

\emph{Methods} are used by the ESPs to balance the electrical grid in the transmission and distribution phases. Examples of methods include regulation, frequency response, grid scale storage and use of renewable sources of energy. Both European and US sites are interested in discussing renewables with their ESPs, but there is little interest in communicating with regards to the other possible methods.

\section{Results: Europe and United States}
\label{res}

%There is moderate interest in grid integration strategies in the United States, and low interest in the same in Europe. From the point of view of SCs, strategies such as cutting jobs or load migration have little interest. \\

%From our questionnaire, we concluded that neither European nor the United States sites are engaged with peak shedding, peak shifting or dynamic pricing programs at present. More sites in the United States have communicated with their ESPs regarding these programs. While both European and United States SCs are interested in dynamic pricing, there is mixed interest in peak shedding and peak shifting. The European sites are more interested in peak shedding than peak shifting, but the United States sites are more interested in peak shifting. \\
%\begin{table}[h]
\begin{center}
\begin{tabular}{|l|c|c|c|c|}
\hline
\multicolumn{5}{|p{.72\textwidth}|}{\emph{Ques:} Please evaluate as high, medium or low the following motivations for your site's interest in pursuing a stronger relationship with your electric service utility/provider}\\
\hline
& Low & Medium & High & Rating Count \\
\hline
Economically justified & 14.3\% (1) & 28.6\% (2) & 57.1\% (4) & 7 \\
\hline
Good citizen & 14.3\% (1) & 71.4\% (5) & 14.3\% (1) & 7 \\
\hline
Adverse consequences & 66.7\% (4) & 16.7\% (1) & 16.7\% (1) & 6 \\
\hline
Government regulation & 71.4\% (5) & 28.6\% (2) & 0.0\% (0) & 7 \\
\hline
\end{tabular}
\end{center}
\caption{Motivation for communicating with ESP (European Respondents)}
\label{fig:table2}
\end{table}

%Both European and US sites are interested in discussing renewables with their ESPs, but there is little interest in communicating with regards to the other possible methods. \\

%We also asked our European respondents to indicate what might motivate them to communicate with their ESPs. The results are shown in Table \ref{fig:table2}. As can be noted from this figure, the main motivators are the financial incentives and the desire to be ``good citizens.''

\begin{figure}[ht!]
\begin{center}
\frame{\includegraphics[scale=0.62]{figs/loadgraphUS.pdf}}
\caption{Total Load at at SCs in United States}
\label{fig:USload}
\vspace{0.9cm}
\frame{\includegraphics[scale=0.62]{figs/loadgraphEU.pdf}}
\caption{Total Load at at SCs in Europe}
\label{fig:EUload}
\end{center}
\end{figure}

We noted that none of the European SCs communicated about grid integration potential (e.g., demand response) and available flexibility with their associated ESPs. Additionally, there was little interest in a tighter integration with the ESPs.

\begin{figure}[ht!]
\begin{center}
\frame{\includegraphics[scale=0.62]{figs/vargraphUS.pdf}}
\caption{Maximum Variability at at SCs in United States}
\label{fig:USvar}
\vspace{0.9cm}
\frame{\includegraphics[scale=0.62]{figs/vargraphEU.pdf}}
\caption{Maximum Variability at at SCs in Europe}
\label{fig:EUvar}
\end{center}
\end{figure}

Figures \ref{fig:USload} and \ref{fig:EUload} depict the total load in megawatts for each of the respondents in the United States and in Europe. Most supercomputing sites have a total load of under 5 MW (sixteen out of twenty). Four of the surveyed supercomputing sites had a total load of over 10 MW. \\

Both United States and Europe had power swings and fluctuations of a few megawatts. In our questionnaire, we asked respondents to report the maximum variability that they have experienced in their SCs. The results of these for United States as well as Europe are shown in Figures \ref{fig:USvar} and \ref{fig:EUvar} respectively. In the United States, three of the eleven sites surveyed had maximum variability of over 5 MW. For our United States respondents, the minimal option for reporting this was ``Less than 3 MW'', because of which we could not capture less intense power swings. In the European survey, we allowed the respondents to provide a more accurate value, and as shown in Figure \ref{fig:EUvar}, we observed power swings in the range of half a megawatt to about 2 MW. Almost all of the respondents reported that this variability is due to maintenance cycles, and that it can be scheduled \emph{day-ahead} if necessary.

\section{Qualitative Analysis: Site-Specific Interviews}
\label{intws}

The results presented in the previous section were based on data gathered through a questionnaire created for HPC centers based on experience from a United States context. The preliminary results of the comparison across the geographical regions gave the impression that European SCs had very limited communication with their ESPs with respect to grid integration. 
However, it was apparent that some SCs in Europe engage in collaboration with their ESPs in order to ensure minimal fluctuations as well as for forecasting of deviations from normal power consumption patterns. 
In order to shed light on the details of the relationships between SCs and ESPs that were not captured in the questionnaire, we designed a qualitative interview and surveyed ORNL, LLNL and LRZ. The thesis was that a qualitative analysis will yield more complete information and will enable us to present more thorough comparative study on the status of grid integration of SCs in Europe and the US. For each site, we asked the questions listed below. We present the information from each SC in the subsections that follow.

\begin{itemize}
\item {What is your responsibility for negotiating the contract between your HPC facility and your ESP? }
\item {Could you elaborate on the details of the pricing structure on your electricity? Note that for this question, we did not request specific information on the actual price the SC pays for electricity. We were mostly interested in the type of pricing program they were enrolled in.} 
\item {Do you have any obligations towards your ESP, and if so, what is your incentive towards committing to these obligations? These obligations are characterized by being static and pre-smart grid, in the sense that no real-time communication is needed between ESP and SC. Examples include limits for allowed variability in power consumption and/or fixed power consumption limits. Examples for potential incentives include reduction in electricity price, enabling of direct payments and legislation benefits. }
\item{Do you offer any kind of services for your ESP, and if so, what is your incentive for offering these services? These services are characterized by two way communication between the site and the ESP, where a consumer reacts to information sent by the ESP. Examples include load capping, powering up backup generations, etc.}
\item{How do you envision your future relationship with your electricity provider? (Possible answers were: tighter, for example, by selling local generation capacity; or looser, for example, by being self-sufficient with regards to electricity needs.}
\end{itemize}

\subsection{Oak Ridge National Laboratory}
For ORNL, DOE negotiates the contract with the ESP.  ORNL gets its power from Tennessee Valley Authority (TVA), which generates, transmits and distributes the power. The DOE and TVA negotiate the power capacity that is being provisioned each year. Typically, a range for operation is chosen, for the current year, this range is 35 MW to 75 MW. 
In terms of electricity pricing, ORNL incurs two kinds of charges: a demand charge, which is fixed for a month, and an energy charge based on actual power consumption. The demand charge is determined by analyzing 30 minute blocks and by determining the peak or maximum value for the month. The demand charge can be off-peak or on-peak based on the time of the day. It also has a time-of-use per day component. ORNL's provider, TVA, is not affected by power swings of a few megawatts (5 to 8 MW) and is very reliable. The goal for ORNL is to keep its HPC systems fully utilized in terms of power. 

ORNL does not have any obligations and provide any services to its ESP.  The only requirement is to operate in the range that was negotiated (35 MW to 75 MW). They have a model that explains their power usage that they provide to the TVA annually, but there is no two-way communication or forecasting. In general, the capital expenditure for the SC at ORNL dominates the operational costs. As the HPC system cost depreciates with time (for example, Titan's depreciation is about 20K dollars per hour), there is little financial incentive to be flexible and to save on electricity costs. The goal is thus to keep their site fully utilized in terms of power. 

\subsection{Lawrence Livermore National Laboratory}
In the case of LLNL,  DOE negotiates the contract with the ESP with the help of a consulting company called Exeter.  A bulk purchase of power is made for about 100 MW of power capacity from the California-Oregon Transmission Project (or COTP) and is shared between LLNL and two other DOE sites. Pacific Gas and Electric (PG\&E) and Western Area Power Administration (WAPA) are used for transmission and distribution. In terms of electricity pricing, LLNL does not pay a demand charge, but only pays a flat energy charge of about 4.5 cents per kWh, which is on the lower side when compared to the industry. Forecasting is done on a regular basis in order to be a good citizen. For the scope of this questionnaire related to the HPC facility, there is not much financial incentive to save energy costs. Additionally, there are no obligations from the ESP and no services are provided. The goal is to keep the site fully utilized in terms of power and to minimize leftover power in order to be energy efficient.  

\subsection{Leibniz Supercomputing Center}
The power contract between LRZ and \emph{Stadtwerke M{\"u}nchen}, a Munich Power Company
is the result of pan-European procurement. LRZ purchases a basic power band for one
or multiple years at the European power stock exchange. Hence, the power price is
determined by the European stock market. Additionally, there are charges for the
power grid, renewable energy, concession levy as well as taxes which are significant.
The charges for power generation and distribution constitute only 25\% of
the power price in Germany. As a result, the energy costs are very expensive for LRZ.

LRZ operates in a 4 to 6 MW power band. Typically, they pay about
17.8 Euro-cents per kWh. Having consistent power consumption is usually considered better,
as huge power swings result in much higher electricity costs. It is thus imperative
to be able to forecast any power swings and to inform the ESP about
the same. Better prediction models for power usage will definitely benefit
LRZ in terms of electricity costs, as one of their goals is to save on
energy costs. This is primarily because their energy costs dominate their
operational costs. Typically, LRZ lets the ESP know about 2 days in
advance for any scheduled downtimes. At present, there are no major
obligations toward or services provided to the ESP, mostly because of
the QOS guarantees that have to be adhered to for their users.

\subsection{Analysis}
The key goals for our qualitative analysis were to understand the power purchase relationships, energy use, and the level of demand management flexibility available to reduce electricity use and/or energy costs for the three SCs under consideration. Our interviews thus focused on the annual electricity purchase negotiations and pricing structure, and on characterizing SC's electricity use relative to larger campus. We also tried to identify the level of motivation for demand management for lowering peak power and energy use and for any services being offered. We observed that while some trends were common across all three sites, there were some differences. We summarize these similarities and differences below. \\

{\bf Similarities:}
An important common trend was that the power purchase negotiations were typically done by a third party (for example, DOE, Exeter or Stadtwerke M{\"u}nchen) and on an annual basis. Power capacity was negotiated by specifying an upper limit on the amount of power procured for all three sites. Additionally, in the case of ORNL and LRZ, a lower bound on the power capacity was also clearly specified. Negotiations for all three sites were done at the level of entire site or a set of collaborative sites, and not merely for the supercomputing facility that was located within the site.\\

{\bf Differences:}
We observed that the pricing structure was different in all three cases. In case of LLNL, there was a flat rate, which makes LLNL less sensitive to electricity cost variation. For ORNL, there was a variable rate, which makes it somewhat sensitive to electricity costs. LRZ, however, is very sensitive to the pricing structure because of the expensive energy costs as well as the impact of power swings on electricity costs. In terms of power fluctuations, LLNL used demand forecasting to be a good citizen. For both LLNL and ORNL, reliability was not a major concern and power variations were acceptable by the ESP. For LRZ, the electricity cost increases if there were more power swings, making them highly responsive to such variability and enabling the need for better forecasting. The electricity generation mix in the United States was mostly thermal, where as in Europe it was largely renewable sources of energy. \\

Overall, we believe that several factors drive the motivation for demand management. The key ones are the control that a site has when it comes to power purchase negotiations, their price sensitivity to power fluctuations, and financial as well as good-citizen-based intentions for communicating their demand with their ESP.  One of the factors that was unclear in this analysis was the contribution of the electricity cost as a part of the site's annual budget or operation costs, which we plan to explore as part of our future work.
\section{Related Work}
\label{relwork}
Data centers are known to be capable of providing flexibility in their power consumption, and thus are great candidates to participate energy market demand response (DR) programs. Wierman et. al.~\cite{WiermanIGCC} survey the opportunities and challenges for data center DR participation. Aikema et. al.~\cite{aikema2012data} overview multiple types of ancillary service markets, and study the capacity and potential benefit by introducing a simple data center participation model. Ghatikar et. al.~\cite{ghatikar2012demand} exploit various load management techniques, e.g., load shedding and shifting for data center DR. Goiri et. al.~\cite{goiri2015matching} propose GreenSlot, a workload scheduler to maximize the green energy consumption (that is, solar energy) while meeting the job deadline. Geographic load migration is another broadly studied data center management technique to help balance the grid, and reduce the energy cost exploiting the electricity price differences~(\cite{wangexploring,wang2013data,chiu2012electric,liu2011greening,lin2012online}). 

The participation of data center in traditional DR programs, such as real-time dynamic energy pricing~(\cite{wang2013sequential,ghamkhari2012data,liu2014pricing}) and peak shaving~(\cite{urgaonkar2011optimal,PSUSigmetrics12,aksanli2013architecting}), has been widely studied. Recently, there are a growing number of interests on the data center participation in emerging DR programs that are more profitable. Chen et. al.~\cite{chenASPDAC} develop real-time dynamic control policies by leveraging both server level power management techniques and server state switches for data centers to provide regulation service reserves (RSRs). They also implement a prototype of the control policies on real-life server clusters with virtualized CPU resource limits~\cite{chendynamic}. Brocanelli et al.~\cite{brocanelli2013joint} propose the joint management of data center and employee Plug-in Hybrid Electric Vehicles (PHEVs) to increase the regulation profit. A systematic comparison shows that RSR is a more profitable program for data centers to participate than traditional programs such as peak shaving~\cite{chenIGCC}. 



\section{Summary and Next Steps}
\label{summary}

In this paper, we conducted a quantitative and qualitative analysis on demand management perspectives in Europe and the United States from the point of view of supercomputing centers with HPC facilities. We surveyed 9 SCs in Europe and 11 SCs in the United States, most of which were part of the Top500 list.  Our key findings were that contrary to our expectation, the SCs in Europe were not communicating actively with their ESPs with regards to demand management approaches. Our qualitative interviews with ORNL, LLNL and LRZ helped us understand the motivation and reasons behind this result. We observe that perspectives on demand management are dependent on the electricity market and pricing in the geographical region and on the degree of control that a particular SC has in terms of power-purchase negotiation.

In summary, we believe that the European ESP programs for DM need to be studied in greater detail and the awareness of the benefits for these programs needs to be raised among the SCs. As part of our future work, we want to explore the European ESP programs further, the lack of such closer relationships, and also conduct a similar study in Japan, which has different institutional and electricity supply challenges. We also want to conduct more qualitative analysis through in-person site interviews to understand the electricity markets and the available incentive better. 


\bibliographystyle{plain}
\bibliography{hao,refs}
\end{document}
