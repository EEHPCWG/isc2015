\section{Summary and Next Steps}
\label{summary}

In this paper, we conducted a quantitative and qualitative analysis on demand management perspectives in Europe and the United States from the point of view of supercomputing centers with HPC facilities. We surveyed 9 SCs in Europe and 11 SCs in the United States, most of which were part of the Top500 list.  Our key findings were that contrary to our expectation, the SCs in Europe were not communicating actively with their ESPs with regards to demand management approaches. Our qualitative interviews with ORNL, LLNL and LRZ helped us understand the motivation and reasons behind this result. We observe that perspectives on demand management are dependent on the electricity market and pricing in the geographical region and on the degree of control that a particular SC has in terms of power-purchase negotiation.

In summary, we believe that the European ESP programs for DM need to be studied in greater detail and the awareness of the benefits for these programs needs to be raised among the SCs. As part of our future work, we want to explore the European ESP programs further, the lack of such closer relationships, and also conduct a similar study in Japan, which has different institutional and electricity supply challenges. We also want to conduct more qualitative analysis through in-person site interviews to understand the electricity markets and the available incentive better. 
