\section{Qualitative Analysis: Site-Specific Interviews}
\label{intws}

The results presented in the previous section were based on data gathered through a questionnaire created for HPC centers based on experience from a United States context. The preliminary results of the comparison across the geographical regions gave the impression that European SCs had very limited communication with their ESPs with respect to grid integration. 
However, it was apparent that some SCs in Europe engage in collaboration with their ESPs in order to ensure minimal fluctuations as well as for forecasting of deviations from normal power consumption patterns. 
In order to shed light on the details of the relationships between SCs and ESPs that were not captured in the questionnaire, we designed a qualitative interview and surveyed ORNL, LLNL and LRZ. The thesis was that a qualitative analysis will yield more complete information and will enable us to present more thorough comparative study on the status of grid integration of SCs in Europe and the US. For each site, we asked the questions listed below. We present the information from each SC in the subsections that follow.

\begin{itemize}
\item {What is your responsibility for negotiating the contract between your HPC facility and your ESP? }
\item {Could you elaborate on the details of the pricing structure on your electricity? Note that for this question, we did not request specific information on the actual price the SC pays for electricity. We were mostly interested in the type of pricing program they are enrolled in.} 
\item {Do you have any obligations towards your ESP, and if so, what is your incentive towards committing to these obligations? These obligations are characterized by being static and ?pre-smart grid? in the sense that no real-time communication is needed between ESP and SC. Examples include limits for allowed variability in power consumption and/or fixed power consumption limits. Examples for potential incentives include reduction in electricity price, enabling of direct payments and legislation benefits. }
\item{Do you offer any kind of services for your ESP, and if so, what is your incentive for offering these services? These services are characterized by two way communication between the site and the ESP, where a consumer reacts to information sent by the ESP. Examples include load capping, powering up backup generations, etc.}
\item{How do you envision your future relationship with your electricity provider? (Possible answers were: tighter, for example, by selling local generation capacity; or looser, for example, by being self-sufficient with regards to electricity needs.}
\end{itemize}


%Table \ref{tabIntw} presents details of this qualitative analysis. We have recorded the results in a tabular form to facilitate comparison between the three sites under consideration. 
%
%\begin{table}
%\centering
%\label{tabIntw} 
%\begin{tabular}{|l|l|l|l|}
% \hline
% \bf{Site} & \bf{ORNL} & \bf{LLNL} & \bf{LRZ} \\ 
% \hline
% HPC System & Titan & Sequoia & SuperMUC \\ 
% \hline
% HPC workload & Basic Science & Basic and & Academic \\
% type &&Nuclear Science&\\
% \hline
% HPC workload & Low & Mission-critical & Low\\
% criticality&&&\\
% \hline
%Point of contact & Jim Rogers & Anna-Maria Bailey & Herbert Huber\\
%\hline
%Responsibility & None, contract & None, contract & None, contract \\
%in negotiating & negotiated by DOE & negotiated by DOE & negotiated at \\
%contract with&&&the site-level by\\
%ESP &&&management\\
%\hline
%Electricity &Demand as well&Only energy & Energy is very\\
%Pricing & as energy charges,&charges, flat&expensive, high power \\
%Structure & depends on season&rate, relatively&taxes as well as\\
%&and time of day. &cheap. Thus care about&season dependent.\\
%&Operate within&utilizing all &Operate within\\
%&a power band&allocated power&power band\\
%\hline
%Power cost & & & \\
%as percentage&About 25-30\%&Under 8\%&About 50\%\\
%of TCO&&&\\
%\hline
%Obligations &&&\\
%Toward ESPs &None, and&None, and&None, and\\
%(and offered & no incentive either& no incentive either&no incentive either\\
%incentives for &&&\\
% the same)&&&\\
%\hline
%Services for &None, except&None, except&None, except\\
%ESPs&maintenance or&maintenance or&maintenance or\\
%&general&general&general\\
%&forecasting&forecasting&forecasting\\
%\hline
%Future &Will use the &Will use the&Will use the \\
%relationship &same, reliable ESP&same, reliable ESP&same ESP, flexibility\\
%with ESP&&&decisions are political.\\
%\hline
% \end{tabular}
% \end{table}

\subsection{ORNL}
For ORNL, the DOE negotiates the contract with the ESP.  ORNL gets its power from TVA, which generates, transmits and distributes the power. The DOE and TVA negotiate the power capacity that is being provisioned each year. Typically, a range for operation is chosen, for the current year, this range is 35 MW to 75 MW. 
In terms of electricity pricing, ORNL incurs two kinds of charges: a demand charge, which is fixed for a month, and an energy charge based on actual power consumption. The demand charge is determined by analyzing 30 minute blocks and by determining the peak or maximum value for the month. The demand charge can be off-peak or on-peak based on the time of the day. It has a time-of-use per day component. The energy charge is a step function. The base charge for energy is under 4 cents per kwh, which is low when compared to the industry. Also, the more energy the site consumes in kWh, the lower is the base charge. As a result, ORNL wants to keep its systems fully utilized in terms of power. Including all the costs as well as seasonal charges, ORNL pays about 9.4 cents per kWh. ORNL's provider, TVA, is not affected by power swings of a few megawatts (5 to 8 MW) and is very reliable. \\

ORNL does not have any obligations and provide any services to its ESP.  The only requirement is to operate in the range that was negotiated (35 MW to 75 MW). They have a model that explains their power usage that they provide to the TVA annually, but there is no two-way communication or forecasting or In general, the capital expenditure for the supercomputing center dominates the operational costs. As the HPC system cost depreciates with time (for example, Titan's depreciation is about 20K dollars per hour), there is little financial incentive to be flexible and to save on electricity costs, as the capital expenditure dominates the operational expenditure. The goal is thus to keep their site fully utilized in terms of power. 

\subsection{LLNL}
In the case of LLNL, the DOE negotiates the contract with the ESP with the help of a consulting company called Exeter.  A bulk purchase of power is made for about 100 MW of power capacity from the California-Oregon Transmission Project or COTP and is shared between LLNL and two other DOE sites. PG\&E is used for transmission and distribution and WAPA is used for generation. In terms of electricity pricing, LLNL does not pay a demand charge, but only pays a flat energy charge of about 4.5 cents per kWh, which is on the lower side when compared to the industry. Forecasting is done on a regular basis to be a good citizen. There is no financial incentive to save energy costs. There are no obligations from the ESP and no services are provided. The goal is to keep the site fully utilized in terms of power and to minimize leftover power.  

\subsection{LRZ}
The power contract between LRZ and \emph{Stadtwerke M�nchen}, a Munich Power Company
is the result of pan-European procurement. LRZ purchases a basic power band for one
or multiple years at the European power stock exchange. Hence, the power price is
determined by the European stock market. Additionally, there are charges for the
power grid, renewable energy, concession levy as well as taxes which are significant.
The charges for power generation and distribution constitute only 25\% of
the power price in Germany. As a result, the energy costs are very expensive for LRZ.

LRZ operates in a 4 to 6 MW power band. Typically, they pay about
17.8 Euro-cents per kWh. Having consistent power consumption is usually considered better,
as huge power swings result in much higher electricity costs. It is thus imperative
to be able to forecast any power swings and to inform the ESP about
the same. Better prediction models for power usage will definitely benefit
LRZ in terms of electricity costs, as one of their goals is to save on
energy costs. This is primarily because their energy costs dominate their
operational costs. Typically, LRZ lets the ESP know about 2 days in
advance for any scheduled downtimes. At present, there are no major
obligations toward or services provided to the ESP, mostly because of
the QOS guarantees that have to be adhered to for their users.

\subsection{Analysis}

