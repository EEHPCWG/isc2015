\section{Site-Specific Interviews}

\section{Discussion}
\label{comm}
From the comments section in our questionnaire, we noted that all SCs are already using \emph{demand forecasting} to communicate their upcoming demands and maintenance cycle schedules with their ESPs. For example, one comment was ``We project hourly average power at least a day in advance, within +/- 1MW''. Another interesting comment was ``We've to ensure that our power load neither over- nor undershoots the contracted power band. In any cases of foreseen power abnormalities we've to inform our grid provider at least two days ahead of schedule.''

One of the SCs mentioned that they could not provide the forecast that was being asked by their ESP. More specifically, their comment indicated that their ESP asked for ``multi-year forecast of energy requirements, additional detailed forecasting and ultimately real time data, and power projections, hour by hour, for at least a day in advance.''

When it came to ESP programs, the United States SCs showed more interest. ``Our site generates 30-35 MW of power yet still imports 5-10 MW. As a large generation source the utility providers see the campus as a highly attractive partner for offloading grid stress. automatic load shedding is being explored/deployed today, '' one of the SCs noted. Another comment was ``[We are] working on load sharing of data with utility to provide better scheduling tools and address potential grid changes.'' One of the SCs mentioned that they demonstrated that peak shedding and shifting was possible, but not deployed due to its impact on HPC productivity. 

The European SCs, on the other had, did not have much knowledge about ESP programs. Some of the responses were ``There are not so many related options and features offered by providers. We are open to further and pro-active efforts as long as providers have other kinds of programs to propose'' and ``With many of your questions I am wondering about the kind of contracts other centers might have and about the quality of some electricity providers.''

The comments also indicated that the SCs in United States are investigating the impact of power fluctuations on the electrical grid. ``[We are] working directly with provider to ensure that the effects of large load swings are understood. Have funded a simulation that accounts for all loads.'' and ``Our provider has no problem with our load swings. They indicate no concern with our next system either, but we are still looking into possible options in case there actually is a problem.'' were some of the interesting responses.
