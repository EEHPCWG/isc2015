\section{Qualitative Analysis: Site-Specific Interviews}
\label{intws}

The results presented in the previous section were based on data gathered through a questionnaire created for HPC centers based on experience from a United States context. The preliminary results of the comparison across the geographical regions gave the impression that European SCs had very limited communication with their ESPs with respect to grid integration. 
However, it was apparent that some SCs in Europe engage in collaboration with their ESPs in order to ensure minimal fluctuations as well as for forecasting of deviations from normal power consumption patterns. 
In order to shed light on the details of the relationships between SCs and ESPs that were not captured in the questionnaire, we designed a qualitative interview and surveyed ORNL, LLNL and LRZ. The thesis was that a qualitative analysis will yield more complete information and will enable us to present more thorough comparative study on the status of grid integration of SCs in Europe and the US. For each site, we asked the questions listed below. We present the information from each SC in the subsections that follow.

\begin{itemize}
\item {What is your responsibility for negotiating the contract between your HPC facility and your ESP? }
\item {Could you elaborate on the details of the pricing structure on your electricity? Note that for this question, we did not request specific information on the actual price the SC pays for electricity. We were mostly interested in the type of pricing program they were enrolled in.} 
\item {Do you have any obligations towards your ESP, and if so, what is your incentive towards committing to these obligations? These obligations are characterized by being static and pre-smart grid, in the sense that no real-time communication is needed between ESP and SC. Examples include limits for allowed variability in power consumption and/or fixed power consumption limits. Examples for potential incentives include reduction in electricity price, enabling of direct payments and legislation benefits. }
\item{Do you offer any kind of services for your ESP, and if so, what is your incentive for offering these services? These services are characterized by two way communication between the site and the ESP, where a consumer reacts to information sent by the ESP. Examples include load capping, powering up backup generations, etc.}
\item{How do you envision your future relationship with your electricity provider? (Possible answers were: tighter, for example, by selling local generation capacity; or looser, for example, by being self-sufficient with regards to electricity needs.}
\end{itemize}

\subsection{Oak Ridge National Laboratory}
For ORNL, DOE negotiates the contract with the ESP.  ORNL gets its power from Tennessee Valley Authority (TVA), which generates, transmits and distributes the power. The DOE and TVA negotiate the power capacity that is being provisioned each year. Typically, a range for operation is chosen, for the current year, this range is 35 MW to 75 MW. 
In terms of electricity pricing, ORNL incurs two kinds of charges: a demand charge, which is fixed for a month, and an energy charge based on actual power consumption. The demand charge is determined by analyzing 30 minute blocks and by determining the peak or maximum value for the month. The demand charge can be off-peak or on-peak based on the time of the day. It also has a time-of-use per day component. ORNL's provider, TVA, is not affected by power swings of a few megawatts (5 to 8 MW) and is very reliable. The goal for ORNL is to keep its HPC systems fully utilized in terms of power. 

ORNL does not have any obligations and provide any services to its ESP.  The only requirement is to operate in the range that was negotiated (35 MW to 75 MW). They have a model that explains their power usage that they provide to the TVA annually, but there is no two-way communication or forecasting. In general, the capital expenditure for the SC at ORNL dominates the operational costs. As the HPC system cost depreciates with time (for example, Titan's depreciation is about 20K dollars per hour), there is little financial incentive to be flexible and to save on electricity costs. The goal is thus to keep their site fully utilized in terms of power. 

\subsection{Lawrence Livermore National Laboratory}
In the case of LLNL,  DOE negotiates the contract with the ESP with the help of a consulting company called Exeter.  A bulk purchase of power is made for about 100 MW of power capacity from the California-Oregon Transmission Project (or COTP) and is shared between LLNL and two other DOE sites. Pacific Gas and Electric (PG\&E) and Western Area Power Administration (WAPA) are used for transmission and distribution. In terms of electricity pricing, LLNL does not pay a demand charge, but only pays a flat energy charge of about 4.5 cents per kWh, which is on the lower side when compared to the industry. Forecasting is done on a regular basis in order to be a good citizen. For the scope of this questionnaire related to the HPC facility, there is not much financial incentive to save energy costs. Additionally, there are no obligations from the ESP and no services are provided. The goal is to keep the site fully utilized in terms of power and to minimize leftover power in order to be energy efficient.  

\subsection{Leibniz Supercomputing Center}
The power contract between LRZ and \emph{Stadtwerke M{\"u}nchen}, a Munich Power Company
is the result of pan-European procurement. LRZ purchases a basic power band for one
or multiple years at the European power stock exchange. Hence, the power price is
determined by the European stock market. Additionally, there are charges for the
power grid, renewable energy, concession levy as well as taxes which are significant.
The charges for power generation and distribution constitute only 25\% of
the power price in Germany. As a result, the energy costs are very expensive for LRZ.

LRZ operates in a 4 to 6 MW power band. Typically, they pay about
17.8 Euro-cents per kWh. Having consistent power consumption is usually considered better,
as huge power swings result in much higher electricity costs. It is thus imperative
to be able to forecast any power swings and to inform the ESP about
the same. Better prediction models for power usage will definitely benefit
LRZ in terms of electricity costs, as one of their goals is to save on
energy costs. This is primarily because their energy costs dominate their
operational costs. Typically, LRZ lets the ESP know about 2 days in
advance for any scheduled downtimes. At present, there are no major
obligations toward or services provided to the ESP, mostly because of
the QOS guarantees that have to be adhered to for their users.

\subsection{Analysis}
The key goals for our qualitative analysis were to understand the power purchase relationships, energy use, and the level of demand management flexibility available to reduce electricity use and/or energy costs for the three SCs under consideration. Our interviews thus focused on the annual electricity purchase negotiations and pricing structure, and on characterizing SC's electricity use relative to larger campus. We also tried to identify the level of motivation for demand management for lowering peak power and energy use and for any services being offered. We observed that while some trends were common across all three sites, there were some differences. We summarize these similarities and differences below. \\

{\bf Similarities:}
An important common trend was that the power purchase negotiations were typically done by a third party (for example, DOE, Exeter or Stadtwerke M{\"u}nchen) and on an annual basis. Power capacity was negotiated by specifying an upper limit on the amount of power procured for all three sites. Additionally, in the case of ORNL and LRZ, a lower bound on the power capacity was also clearly specified. Negotiations for all three sites were done at the level of entire site or a set of collaborative sites, and not merely for the supercomputing facility that was located within the site.\\

{\bf Differences:}
We observed that the pricing structure was different in all three cases. In case of LLNL, there was a flat rate, which makes LLNL less sensitive to electricity cost variation. For ORNL, there was a variable rate, which makes it somewhat sensitive to electricity costs. LRZ, however, is very sensitive to the pricing structure because of the expensive energy costs as well as the impact of power swings on electricity costs. In terms of power fluctuations, LLNL used demand forecasting to be a good citizen. For both LLNL and ORNL, reliability was not a major concern and power variations were acceptable by the ESP. For LRZ, the electricity cost increases if there were more power swings, making them highly responsive to such variability and enabling the need for better forecasting. The electricity generation mix in the United States was mostly thermal, where as in Europe it was largely renewable sources of energy. \\

Overall, we believe that several factors drive the motivation for demand management. The key ones are the control that a site has when it comes to power purchase negotiations, their price sensitivity to power fluctuations, and financial as well as good-citizen-based intentions for communicating their demand with their ESP.  One of the factors that was unclear in this analysis was the contribution of the electricity cost as a part of the site's annual budget or operation costs, which we plan to explore as part of our future work.